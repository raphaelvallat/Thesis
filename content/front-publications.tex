\addchap{List of publications}
\label{sec:publications}
\vspace*{-10mm}

\textbf{Peer-reviewed publications}

\underline{Vallat R.}, Lajnef T., Eichenlaub J.-B., Berthomier C., Jerbi K., Morlet D., and Ruby P. (2017). Increased Evoked Potentials to Arousing Auditory Stimuli during Sleep: Implication for the Understanding of Dream Recall. \href{https://doi.org/10.3389/fnhum.2017.00132}{Frontiers in Human Neuroscience}, 11.

\underline{Vallat R.*}, Combrisson E.*, Eichenlaub J-B., O'Reilly C., Lajnef T., Guillot A., Jerbi K. and Ruby P. (2017). Sleep: an open-source python software for visualization, analysis and staging of sleep data. \href{https://doi.org/10.3389/fninf.2017.00060}{Frontiers in Neuroinformatics}, 11. - \emph{* Co-first authors}

\underline{Vallat R.}, Chatard B., Blagrove M. and Ruby P. (2017) Characteristics of the memory sources of dreams: a new version
of the content-matching paradigm to take mundane and remote memories into account. \href{https://doi.org/10.1371/journal.pone.0185262}{Plos One}, 12. 

\textbf{Under review}

\underline{Vallat R.}, Meunier D., Nicolas A. and Ruby P. Reduced default mode network connectivity and anti-correlation in the minutes following awakening from N2 and N3 sleep: an EEG-fMRI study.

\underline{Vallat R.}, Eskinazi M., Nicolas A. and Ruby P. Sleep habits and dream recall frequency in a representative sample
of French students.

\textbf{In preparation}

\underline{Vallat R.}, Nicolas A. and Ruby P. Brain functional connectivity upon awakening from sleep predicts between-subject differences in dream recall frequency.

Combrisson E., \underline{Vallat R.}, O'Reilly C., Pascarella A., Saive A-L., Thiery T., Meunier D., Althukov D., Lajnef T., Ruby P., Guillot A. and Jerbi K. Visbrain: A multi-purpose GPU-accelerated open-source suite for brain data visualization.

Ruby P., Chatard B., \underline{Vallat R.}, Hoyer R. and Bidet-Caulet A. Top-down and bottom-up attentional processes in high and low dream recallers: an EEG study.

Plailly J., Villalba M., \underline{Vallat R.}, Nicolas A. and Ruby P. Recalling a dream related to a recent experience: does it help episodic memory consolidation?
