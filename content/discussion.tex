\cleardoublepage
\chapter{Neurophysiological and behavioral factors associated with a high DRF}
\label{disc:drf}

\section{Summary of the results}
\label{disc:drf:summary}

One of the major objectives of the present thesis was to investigate the neurophysiological and behavioral correlates of inter-individual variability in dream recall frequency (DRF). Based on previous findings, we hypothesized that DRF is associated with a specific psychological and physiological functioning during both sleep and wakefulness. To test this hypothesis, we conducted several experiments to compare the brain activity, sleep parameters, cognitive abilities and personality traits of high and low dream recallers (HR and LR, respectively). The main findings of our experiments are summarized in Fig \ref{fig:disc:drf:summary}.

\begin{figure}[!htb]
	\includegraphics[width=\textwidth]{Fig/Discussion/HR_recap.png}
	\caption[Summary of the results on DRF]{\textbf{Summary of the differences observed between high and low dream recallers}. The findings of the present thesis are written in bold. By opposition, DRF was not associated with memory abilities or the density of certain sleep microstructural events such as rapid eye movements, muscle twitches, spindles and K-complexes.}
	\label{fig:disc:drf:summary}
\end{figure}

\subsection{DRF is positively associated with increased brain reactivity and longer intra-sleep awakening}
\label{disc:drf:summary:arousals}

In Study 1, we performed an in-depth investigation of the sleep macro and micro-structure of HR and LR by re-analyzing the polysomnographic recordings of \citet{eichenlaub_brain_2014}. First, we did not find any significant between-group differences in any of the sleep microstructural features considered (e.g. arousals, spindles, K-complexes, REMs). Our interpretation of these findings is that, most probably, sleep microstructural features are not crucial factors to explain DRF variability.
By contrast, we observed that full awakenings (i.e > 15 seconds) were longer in all sleep stages in HR as compared to LR (roughly 2 vs 1 min respectively). Noteworthy, the number of awakenings was not different between the two groups.

These observations led us to propose that among sleep parameters, the duration of intra-sleep wakefulness seems to be the most critical predictor of inter-individual differences in DRF. This result provides a strong evidence in favor of the arousal-retrieval model (see section \ref{sec:dream-recall:theories:arousal}, \citealp{koulack_dream_1976}), which states that a short period of wakefulness has to occur immediately after dreaming in order to transfer the dream content from short to long term memory. Our findings constitute an important contribution to this model by demonstrating, using objective measurements, a link between intra-sleep awakening and DRF. These results also extended this model by showing that the duration of awakenings is more critical to dream recall than the number of awakenings. We proposed that awakenings must be of sufficient duration to allow successful encoding of dreams into memory. Based on our results and previous ones \citep{campbell_perception_1981}, we suggested that 2 minutes might be the threshold duration for a successful encoding.

On this point, it should be noted that the author of the present thesis contributed to an ongoing study aiming at investigating, by means of human intra-cortical EEG, the temporal dynamic of reactivation of brain regions involved in memory processing during arousals (ranging from 3 sec to 2 minutes). Preliminary results showed that the spectral composition of hippocampal EEG signal during these arousals was intermediate between that of sleep and wakefulness activities in NREM and REM sleep, and that this activation was modulated by the awakening duration (Eskinazi et al., \emph{in preparation}, see \hyperref[sec:publications]{Publications} list). Furthermore, we observed that hippocampus activity during these arousals was different during NREM and REM sleep, a finding particularly relevant considering the well-known dichotomy between these two sleep stages with regards to dream recall and, more broadly, memorization processes \citep{nielsen_review_2000, conduit_poor_2004}.

The visual scoring of arousals allowed us to address another issue, which is related to the finding of differential brain reactivity to auditory stimuli in high and low dream recallers (see section \ref{sec:dream-recall:param:neuro}). \citet{eichenlaub_brain_2014} has suggested that there might be a causal link between the larger brain responses to auditory stimuli and greater intra-sleep wakefulness during sleep in HR as compared to LR. In other words, the amplitude of brain responses to auditory stimuli could be predictive of subsequent awakening or arousal reactions, an observation that has been previously reported for nociceptive stimuli \citet{bastuji_laser_2008}. To test this hypothesis, we computed the auditory evoked potentials to arousing stimuli (i.e. inducing either an arousal or awakening within the next 15 sec) or non-arousing stimuli (i.e. stimuli that do not induce a disruption of the PSG signal within the next 15 sec). This comparison was not possible without the tedious and time-consuming visual scoring of arousals, given that arousals are far more frequent than awakenings in a normal night of sleep, and are therefore needed to compute reliable and statistically valid evoked potentials. Consistent with our hypothesis, we have shown that brain responses to auditory stimuli, in N2 sleep, were larger when followed by a subsequent arousing reaction. Importantly, this increase in the amplitude of the brain responses seemed to be truly related to the stimulus since the amplitude was larger when arousing reactions were within 5 seconds after the stimulus compared to when they were between 5 and 15 seconds. Although it was not possible to compare the brain responses to arousing stimuli between HR and LR (because of too few subjects in each group having a sufficient number of arousing stimuli), behavioral results showed that HR elicited a significantly greater proportion of arousing reaction than LR, thus confirming the idea of a greater brain reactivity to external stimuli in HR \citep{eichenlaub_brain_2014}. It is important to note that these findings may also suggest differential attentional processes between HR and LR. This hypothesis was recently tested in an EEG study in which the author of the present thesis contributed (Ruby et al., \emph{in preparation}, see \hyperref[sec:publications]{Publications} list). Preliminary findings showed an increase of both top-down and bottom-up attentional processes in HR and LR during wakefulness, thus reinforcing the idea of a differential cognitive and brain functioning between the two groups.

In sum, our findings argue for the existence of a causal link between intra-sleep awakening and the brain reactivity to stimuli during sleep. As compared to LR, a greater brain reactivity during sleep in HR could promote intra-sleep awakening which could in turn promote dream recall. Our hypothesis is that if these awakenings are of sufficient duration to allow for the reactivation of the memory encoding abilities of the brain (and notably the hippocampus), then the dream content can be successfully encoded into long-term memory and therefore successfully recalled in the morning. That said, it should be taken into account, however, that DRF variability is unlikely to be explained fully through this one mechanism, since studies have shown that even when awakened at specific moment during the night and under controlled laboratory condition, LR still report significantly less dreams than HR (\citealp{goodenough_comparison_1959}, also replicated in Study 2 of this thesis). Consequently, it is reasonable to assume that some other factors mediate the forgetting and recalling of dreams. A factor that has been proposed but surprisingly never experimentally tested until now is the brain and cognitive functioning during the transition from sleep to wakefulness (i.e. sleep inertia).

\subsection{DRF is positively associated with brain functional connectivity upon awakening}
\label{disc:drf:summary:inertia}

In Study 2, we tested the hypothesis of a differential sleep inertia between HR and LR. To this aim, we designed an EEG-fMRI sleep study to compare specifically the brain functional connectivity and cognitive performances of these two groups following awakening from a daytime nap. To our knowledge, this was the first study to experimentally test the relationship between sleep inertia and DRF. Our predictions were that HR would show less cognitive impairments and brain functional alterations than LR at awakening, therefore allowing them to better encode dream content upon awakening from sleep.

While we were not able to evidence significant behavioral between-group differences at awakening (discussed in section \ref{res:inertia:drf:discussion}), our results showed on a differential brain functional organization associated with DRF in the minutes following awakening from sleep. We found that at 5 min-post-awakening, HR exhibited a greater functional connectivity within the default mode network and regions involved in memory retrieval, such as the medial prefrontal cortex (MPFC), the precuneus, the left medial temporal lobe (MTL) and the left dorsolateral prefrontal cortex (DLPFC). Remarkably, these are almost exactly the same regions found to be involved in episodic memory encoding and retrieval (reviewed in \citealp{spaniol_event-related_2009}). Our interpretation of these results is that the higher functional connectivity in mnemonic brain regions observed in HR could facilitate in these participants the retrieval of dream content, by preventing the loss of the short-term dream memory during the sleep-wake transition. Inversely, LR could fail to recall their dreams because of greater functional connectivity alterations during the first minutes following awakening. More broadly, our results argue in favor of a differential functional awakening process between HR and LR that could explain inter-group differences in dream recall.

On another topic, it is important to note that this study was also the first to investigate simultaneously the brain and cognitive alterations of sleep inertia in healthy subjects (\hyperref[res:inertia:inertia]{part 1 of Study 2}). Using measures of arithmetic performances at pre-sleep, 5 min and 25 min post-awakening, we replicated the finding of reduced cognitive performances just after awakening as compared to before sleep or 25 min after awakening. Furthermore, we provided a brain mechanism for these cognitive impairments, by showing a global loss of brain functional \emph{segregation} following awakening from N2 and N3 sleep. Consistent with the well-known link between the severity of sleep inertia and the prior sleep stage \citep{tassi_sleep_2000}, we found that awakening from N3 sleep was associated with the most severe and robust changes in the brain functional connectivity. Among the perspectives for future studies, it would be interesting to extend these data to N1 sleep and REM sleep, which are known to induce less sleep inertia than N2 or N3 sleep. However, REM sleep is very difficult to observe in an MRI setting, unless applying a severe and specific REM sleep deprivation in the night(s) before \citep{duyn_eeg-fmri_2012}, which is of course not ideal to study functional connectivity given the huge impact of severe sleep deprivation on the brain functional connectome \citep{de_havas_sleep_2012, yeo_functional_2015, krause_sleep-deprived_2017}.

\subsection{DRF is positively associated with creative-thinking abilities and default mode network connectivity}
\label{disc:drf:summary:dmn}

There is a rising consensus that dreaming, or at least dream recall, could be subserved by regions of the default mode network (DMN). In Study 3, we re-analyzed the fMRI data of Study 2 to specifically investigate the relationship between DRF and the DMN. Our results show that, during rest and compared to LR, HR exhibit a higher functional connectivity within the DMN (1) in average and (2) specifically between the MPFC and TPJ. These results are remarkably consistent with previous ones showing a higher rCBF in HR between these two same regions during sleep and wakefulness \citep{eichenlaub_resting_2014}, and a cessation of dream reporting following focal lesions in these brain areas \citep{solms_neuropsychology_1997}. Based on all these observations, one can reasonably argue that the TPJ and the MPFC are two critical regions when it comes to the ability to recall dreams. The question remains yet pending whether these regions are only involved in dream recall during wakefulness, or also in the production of dreams during sleep. It would be premature to answer that question given that we still have no other means than awakening the participants to assess whether he or she was dreaming.

The second goal of this study was to compare the cognitive abilities (e.g. memory, creativity) and personality traits of HR and LR. We found that HR scored higher than LR on measures of creative-idea generation, without any further between group differences in memory or cognitive abilities. Regarding personality traits, we found that HR tended to score higher on several big five dimensions such as neuroticism, agreeableness and openness-to-experience. These differences were however not significant. This could be due, in part, to the number of participants (n=55), which despite being great for a typical neuroimaging study (especially involving simultaneous EEG-fMRI recordings), is rather low for behaviorally assessing subtle differences in personality traits (e.g. n=981 in \citealp{hartmann_boundaries_1989}). The finding of a higher creativity in HR than in LR, which has already been reported in several studies \citep{fitch_variations_1989, schredl_creativity_1995, schredl_factors_2003}, is particularly interesting given that creative-thinking has also been associated with the recruitment of the DMN \citep{ellamil_evaluative_2012, jung_structure_2013, beaty_creativity_2014, mok_interplay_2014, beaty_default_2015, christoff_mind-wandering_2016}. As such, these findings are consistent with the emerging view that creative-thinking and dreaming share some phenomenological and neurophysiological properties \citep{christoff_mind-wandering_2016}.

Altogether, these results argue in favor of Schonbar's claim \citeyearpar{schonbar_differential_1965} that high or low DRF can be explained by the \q{life-style} of individuals (among which are creative-thinking abilities and personality traits). Our findings go one step further by suggesting that this life-style is related to a specific brain functioning, characterized notably by an increased functional connectivity in the DMN. As we will discuss in section \ref{disc:drf:model}, the question remains to whether there is a causal link between all these variables, and notably whether personality traits, life-style and DRF variations can significantly influence and modify the brain functional properties (and reciprocally).

\subsection{DRF is associated with age, gender, and clarity of dream content}
\label{disc:drf:summary:survey}

We took advantage of the recruitment questionnaire of the EEG-fMRI study on sleep inertia to perform an epidemiological survey of the sleep and dream habits of a large sample of French college students from Lyon University. The survey included several questions regarding DRF. Remarkably, we were able to evidence a negative correlation between DRF and age, even on the tight age range of our sample (i.e. from 18 to 30 years old), as well as a positive correlation between DRF and the clarity of dreams. Furthermore, we were able to replicate the finding of a higher DRF in women than in men \citep{schredl_gender_2008}. Many factors could explain, at least partly, this gender difference in DRF. For instance, \citet{schredl_gender_2008} proposed that it was the result of a gender-specific dream socialization process during childhood. According to them, girls are encouraged more often than boys to talk about their dreams during their childhood, and might therefore develop a stronger interest in dreams, a factor consistently reported to be positively associated with DRF \citep{schredl_factors_2003}. Another possible explanation could be the higher proportion of intra-sleep wakefulness reported in women as compared to men \citep{reyner_gender-and_1995}, which could give them more opportunities to encode dreams into memory according to the arousal retrieval-model \citep{koulack_dream_1976}. Finally, drawing from Schonbar's life-style hypothesis \citeyearpar{schonbar_differential_1965}, one can argue that the higher DRF in women could be the result of differential personality and cognitive traits between men and women. This observation is supported by studies showing higher levels of neuroticism, extraversion, agreeableness, and conscientiousness in women compared to men, as well as a tendency for higher creativity (reviewed in \citealp{schmitt_why_2009, baer_gender_2008}).

Remarkably, age, gender and clarity can all be related to a specific functioning of the default mode network (DMN). For instance, women tend to exhibit a higher functional connectivity in the DMN \citep{bluhm_default_2008}. Similarly, in comparison with younger subjects, older people were consistently found to exhibit a global lower functional connectivity within the DMN \citep{damoiseaux_reduced_2008, koch_effects_2010}, a finding well in line with the observation of reduced mind-wandering abilities in older people \citep{jackson_mind-wandering_2012}. Finally, there is a rising consensus that DMN may be involved in visual imagery process \citep{andrews-hanna_functional-anatomic_2010}, thus suggesting that a high DMN activity could be causally linked to an increased clarity of the dream content. By extension, this could mean that DMN activity is directly related to the salience of dream content. We will discuss further these interactions between DMN functioning and other factors related to DRF in the next section, in which we propose an integrative model of dream recall based on all these findings.

\section{An integrative model of dream recall}
\label{disc:drf:model}

How can we combine the above findings on DRF variability to the previously existing knowledge on dream recall?
The results of the present thesis led us to propose a comprehensive and integrative model of dream recall, depicted in Fig \ref{fig:disc:drf:model}.

The main assumption of this model is that successful dream recall requires two successive steps, namely the \emph{survival} of the dream content during the sleep wake-transition and the \emph{encoding} of the dream content from short to long term memory. As such, our model draws from the arousal-retrieval model \citep{koulack_dream_1976}, which states that the encoding of dream content into long-term memory is not possible during sleep but only during wakefulness. This model has received support from several experimental studies, including the Study 1 of the present thesis in which we observed a positive association between DRF and the duration of intra-sleep awakenings. Our results led us to propose that there might be a threshold duration to allow the full reactivation of the memory encoding abilities of the brain, which we estimated to be around two minutes. Furthermore, our ERPs results suggest that intra-sleep awakenings might be causally linked to the brain reactivity to auditory stimuli. Based on these findings, we propose that the encoding of the dream content from short to long term memory is dependent of the duration of intra-sleep awakenings, which is in turn linked to the processing of external stimuli during sleep. In addition, our model incorporates, on the psychological side, the interference hypothesis of \citet{cohen_dream_1973}, who proposed that the dream memory trace remains so long as there is no distraction or interferences in the encoding process.

Second, we postulate that in order to be successfully encoded into a long-term memory, the dream content first needs to survive the sleep-wake transition. Indeed, we have seen in the Study 2 of the present thesis that the brain undergoes dramatic changes during the transition from sleep to wakefulness. We found notably that awakening from N3 sleep induces more severe brain functional alterations than awakening from N2 sleep, thus suggesting a causal link between sleep depth and the severity of the functional alterations at awakening. With regards to dream recall, we observed a higher functional connectivity in HR in the first minutes following awakening. This result suggests that the severity of brain alterations at awakening is causally linked to dream recall, an idea well in line \citet{koukkou_dreaming:_1983}'s stage-shift hypothesis according to which the forgetting of dreams is a function of the magnitude of the difference between the pre- and post-awakening brain state. Given that awakening from N3 sleep induces the most severe changes in the brain functional organization, it is not surprising therefore that awakening from N3 sleep has been consistently associated with the lowest frequency of dream recall. Another, more psychological, factor that, according to our model, plays a crucial role in the survival of the dream content during the sleep-wake transition is the salience of the dream content. This idea was first proposed by \citet{cohen_test_1974} and received further support from experimental studies since then \citep{cipolli_bizarreness_1993, schredl_emotions_1998}. Put it simply, this hypothesis states that the more salient a dream is (e.g. vivid, bizarre, emotionally intense), the more likely it will be recalled. In sum, we propose that two conditions are necessary for dream content to survive the sleep-wake transition. First, the awakening must preferentially occur in a brain state functionally close to wakefulness (typically REM sleep), in order to limit the magnitude of the brain functional alterations upon awakening. Second, the dream content must contain salient features.

Drawing from \citet{schonbar_differential_1965}'s life-style hypothesis, we further propose that all these state factors might be influenced by psychological and cognitive traits factors. For instance, the level of interference during the encoding process might be related to the interest in dreams, which has been consistently found to be positively associated with DRF \citep{schredl_factors_2003}. One explanation could be that individuals highly interested in their dreams might make a voluntarily effort upon awakening to \emph{grasp} the dream memory and consequently reduce interferences in the encoding process. This mechanism could also explain why DRF is known to be significantly enhanced by keeping a dream diary \citep{schredl_questionnaires_2002}. Similarly, the salience of dream content might also be related to some traits factors, including creative-thinking abilities, life-style, and personality traits.

One of the key finding of the present thesis is that this differential psychological profile between high and low dream recallers is also associated with a specific neurophysiological profile, characterized notably by a higher functioning of the default mode network (DMN) in high dream recallers. According to our model, this neurophysiological profile could be interdependently linked with psychological parameters, and exerts as such an influence on all the state factors involved in the process of dream recall. For instance, a higher DMN functioning could be related to creative-thinking abilities, which could determine in turn the salience of dream content. Second, higher DMN functioning in HR during sleep could reduce brain functional alterations at awakening, therefore facilitating the survival of dream content during the sleep-wake transition, and at the same time promote the brain reactivity to external (and internal) stimuli, leading to increased intra-sleep wakefulness and thus more opportunities to encode dream content into memory. Lastly, age and gender have been both related to changes in DRF (see Study 4), and changes in psychological and neurophysiological traits. Our model propose therefore that age and gender could mediate several factors related to the dream recall process. For instance, reduced cognitive and DMN functioning in older people could prevent a successful encoding of dream recall into memory and explain the well-known negative correlations between age and DRF. Similarly, specific personality traits in women, associated with a higher DMN functioning, could result in an increased salience of dream content and greater intra-sleep wakefulness, therefore explaining the small but consistent gender effect found in DRF.

\begin{figure}[!htbp]
	\includegraphics[width=\textwidth]{Fig/Discussion/schema_dream_recall.png}
	\caption[An integrative model of dream recall]{An integrative model of dream recall.}
	\label{fig:disc:drf:model}
\end{figure}

\section{Conclusions and perspectives}
\label{disc:drf:perspectives}

In summary, the different studies on DRF of the present thesis improved our knowledge of the factors and their interactions contributing to the process of dream recall. Based on the latests experimental findings, we proposed an integrative model of dream recall which can hopefully serve as a basis for future work. Among possible future axes of research, it seems promising to test whether DMN activity could also predict intra-individual variability in DRF across time. To this aim, one could use DRF enhancing methods (such as keeping a dream diary; \citealp{schredl_questionnaires_2002}) to test whether an increased DRF would result in increased creativity scores and DMN functional connectivity in post compared to pre-training measures within the same individuals (preferentially an initial group of low dream recallers).

%%%%%%%%%%%%%%%%%%%%%%%%%%%%%%%%%%%%%%%%%%%%%%%%%%%%%%%%%%%%%%%%%%%%%%%%%%%%%%%
\cleardoublepage
\chapter{The relationship between waking life and dream content}
\label{disc:wle}

\section{Such stuff as dreams are made on}
\label{disc:drf:summary:residue}

\myepigraph{We are such stuff,\\ As dreams are made on; and our little life, \\ Is rounded with a sleep}{William Shakespeare}{The Tempest. 1611}

Through an extensive investigation of the relationship between waking-life and dream content, Study 5 significantly improved our knowledge of the \emph{stuff that dreams are made on}. We asked participants to record and describe, over a period of one week, the obvious connections that they could make between their waking-life and dream content. By specifically investigating the characteristics of waking-life experiences (WLE) incorporated into dreams, we enhanced our understanding of the filter that dreaming applies to waking life.

We observed that the \q{dream mixture}, as Freud called it, is composed of several types of WLE which are all incorporated in significant proportions, i.e. recent and old, emotionally loaded and emotionless, positive and negative, important and insignificant, concerns and non-concern issues, to name but a few. Remarkably, we also found a significant interaction between the temporal remoteness of WLE and their emotional intensity. Older memories were scored by the subjects as the most emotionally intense and important, by opposition with memories of the day before (i.e. day-residues) that were mainly self-rated as non-important and emotionally neutral. However, it should be noted that this effect could be partly explained by the generally low-frequency of emotionally intense WLE. Indeed, none of the participants experienced a highly emotional experience during the 7 days of the experiment. Taken together, the observations of this study led us to support \citet{payne_sleep_2004}'s claim that dream content reflects certain memory processes taking place during sleep. Notably, the selective consolidation and/or forgetting of new memories during sleep (i.e. \q{memory triage}, \citealp{stickgold_sleep-dependent_2013}) could be function of the adaptive relevance and the emotional intensity of these memories \citep{schwartz_are_2003, malinowski_memory_2014, saletin_role_2011}.

\section{A role of dreaming in emotional regulation}
\label{disc:drf:summary:regulation}

The questionnaires that the participants had to fill in each morning after awakening included questions regarding the emotional tone of the waking-life memories, not only as they were experienced originally, but also as they were experienced within the dream content. Remarkably, we found that the dreamed version of the WLE was emotionally down-regulated compared to its waking-life counterpart form. Both emotionally positive and negative WLE were rated as less emotionally intense within dreams as compared to their original occurrence in waking-life.

These results suggest the existence of a down-regulation of emotional waking memories during dreaming (i.e. attenuation of the emotional intensity of waking memories toward a more neutral tone), and provide as such one of the very few experimental evidences supporting the emotional regulation theory of dreaming \citep{cartwright_role_1998, cartwright_role_1998-1, perogamvros_roles_2012}, which claims that dreaming may actively moderate mood overnight in healthy individuals (see section \ref{sec:dream-func:modern:emotion}). Taking up the idea of dreams as an open-window on the cognitive processes occurring during sleep, this down-regulation of emotional waking memories observed in dream content could be the result of an overnight modulation of affective neural system and reprocessing of emotional experiences \citep{walker_overnight_2009, goldstein_role_2014}. Noteworthy, well in line with our observations, a recent ERP study suggested a dissociation between the informational and emotional components of memories during REM sleep, which might, according to the author, result in a strengthening of the informational core of the memories combined with a reduction of the affective tone \citep{groch_role_2013}. Furthermore, in addition with being involved in mood regulation, this recombination of memories during sleep may also lead to creative insights and new ideas \citep{maquet_psychology:_2004, payne_sleep_2004, edwards_dreaming_2013, barrett_dreams_2017}. All the more reason, then, to believe that Hamlet was right to say \q{to sleep, perchance to dream} (Shakespeare, Hamlet, 1603).

\section{Perspectives}
\label{disc:drf:summary:perspectives}

Several open questions remain following this work. For instance, several studies demonstrated that time of night affects wake–dream continuity \citep{roffwarg_effects_1978, malinowski_effect_2014}, with notably a preferential incorporation of memories from the recent past in the beginning of the night, and a preferential incorporation of memories from the distant past in the end of the night. This suggests the thought-provoking idea that consolidation and regulation processes of waking memories follows a sequential pattern throughout the night. Accordingly, novel and salient waking memories from the recent past could be prioritized and processes earlier in the night than old memories. It would be thus interesting to test, using our protocol, whether we could find differences between the characteristics of the WLE observed from spontaneous awakening (i.e. at the end of the night) and those of the WLE observed earlier in the night (for example, by asking participants to put an alarm clock 2 or 3 hours after going to bed).

Another, perhaps more theoretical issue, relates to whether the dream does really incorporate both day-residues and old memories, or rather that these old memories are somehow linked to day-residues. This latter idea was proposed by \citet{freud_interpretation_1900} who noticed that \q{references to earlier episodes in life may also be incorporated [into dreams], but these episodes were always linked somehow to the dream-day and were therefore, day-residues. For example, they could have been recalled during the dream-day or perhaps reflect the same concern as the day-residue} \citep{marquardt_empirical_1996}. This problem has also been raised by \citet{grenier_temporal_2005} who proposed that \q{it would be interesting to examine the profile of references [i.e. WLE] that were identified as having been recently thought of or talked about, and to trace the life period to which they refer in terms of the last time seen or experienced in reality}. The issue remains, however, as to whether the participants would be able to remember all their daily thoughts, words and deeds.

Finally, it should be added that the author of the present thesis contributed to a study which aimed at investigating the putative role of dreaming in memory consolidation (see section \ref{sec:dream-func:modern:memory}). To this aim, we tested whether recalling a dream related to a recent experience is associated with improved post-sleep memory performance (Plailly et al., \emph{in preparation}, see \hyperref[sec:publications]{Publications} list), using an ecological non-explicit visuo-olfactory learning task \citep{saive_novel_2013}. Participants were presented with a visuo-olfactory environment (odors presented at precise locations of a landscape image) during 7 minutes for 3 consecutive days. They were also asked to record their dreams during the 3 nights following the learning (subjects were selected as high dream recallers). Memory for the multi-sensory episodes was tested on the fourth day of the experiment. Both between-subjects and intra-subjects comparisons revealed no significant effect of dream content on odor recognition and episodic retrieval. In other words, we found no significant effect of the incorporation of the learning phase into dream reports on memory performance. Our results therefore do not argue for the hypothesis of a link between the incorporation of a task into dream report and the subsequent memory of this task. As pointed out by \citet{schredl_is_2017}, \q{the research in this area is, however, just at its beginning}, and further studies are needed to either replicate or refute these findings.

%%%%%%%%%%%%%%%%%%%%%%%%%%%%%%%%%%%%%%%%%%%%%%%%%%%%%%%%%%%%%%%%%%%%%%%%%%%%%%%
\cleardoublepage
\chapter{Methodological development}
\label{disc:methods}

\section{A state-of-the-art open-source software}
\label{disc:methods:software}

SLEEP is a free, cross-platform and open-source graphical user interface dedicated to sleep reading, scoring and analysis. Initially designed for a personal use, it soon extended into a fully developed and comprehensive software thanks to a close collaboration with a fellow PhD student, Etienne Combrisson. SLEEP has many advantages over other existing solutions. First, and perhaps most importantly, it is free and open-source. Second, it leverages the graphics processing unit to deliver cutting edge graphical performances. Third, it natively supports several commercial and public data file formats, thus making it accessible to the greatest possible number of people. Fourth, it implements several signal processing tools, as well as several automatic detections of sleep microstructural features. Fifth, it comes with an extensive documentation, a chat room and a peer-reviewed publication. In view of all these functionalities, one can reasonably conclude that SLEEP represents a state-of-the-art software in sleep research which should consequently benefit many. Furthermore, as the development of the software is still ongoing, novel functionalities will continue to be added. Some of these future perspectives are detailed in the section below.

\section{Future directions}
\label{disc:methods:future}

SLEEP includes so far 5 algorithms for detecting some of the most prominent features of each sleep stage, namely spindles, K-complexes, slow waves, rapid eye movements and muscle twitches. Two of these detections (spindles and K-complexes) were compared against a visual scoring reference and showed overall good performances. Yet, there are still opportunities for further enhancements and the detection algorithms were improved since the initial, published, version. An example of the updated spindles detection pipeline can be found in Fig \ref{fig:disc:methods:future:spindles}.

\begin{figure}[htb]
	\includegraphics[width=\textwidth]{Fig/Discussion/spindles.png}
	\caption[Improved spindles detection algorithm]{\textbf{Improved spindles detection algorithm.} Compared to the initial algorithm (presented in chapter \ref{res:software}), the new spindles detection has several improvements. First, we implemented a data-driven tuning of the spindle frequency band by finding the peak spectral power within the sigma range. This step, described by \citet{berthomier_automatic_2007}, allows to accommodate for inter-individual variability of EEG signals, and is particularly useful when analyzing patients who tend to exhibit higher variability. Second, we now use both a hard and a soft threshold on the amplitude of the wavelet transform to determine more precisely the beginning and the end of each spindle. Finally, to allow users to better understand how the detection algorithm works, we implemented a function to plot the current figure for each desired time window.}
	\label{fig:disc:methods:future:spindles}
\end{figure}

Furthermore, perhaps one of the most challenging issue in sleep research is the scoring of sleep stages. Currently, the gold standard remains visual scoring by an expert, which is time-consuming and subject to high inter-rater variability. There is therefore a crucial need for reliable and time-efficient algorithms capable of detecting sleep stages in healthy and patients alike. With this in mind, we are currently working on two distinct automatic sleep scoring methods, based respectively on spectral feature extraction / microstructural detection (Fig \ref{fig:disc:methods:future:autoscore}), and on machine-learning algorithms. Preliminary results based on the former method show a 81\% agreement with a manually scored standard reference, a figure comprised within the range of human inter-scorer agreements (generally between 80 and 90\%, see \citealp{silber_visual_2007}). A similar agreement was obtained using the second, machine-learning based automatic sleep scoring method. Future developments will be needed to get the most out of these two methods and ultimately provide a state-of-the-art algorithm.

\begin{figure}[htb]
	\includegraphics[width=\textwidth]{Fig/Discussion/autoscore.png}
	\caption[Preliminary results of the automatic sleep scoring algorithm]{\textbf{Preliminary results of the automatic sleep scoring algorithm.} The algorithm is based on a combination of spectral features extraction (Top) and automatic detection of microstructural features (e.g. spindles, blue dots). Preliminary tests on a single EEG channel (C3) of one healthy individual yielded 81\% agreement between the predicted and the manually scored full night hypnogram (with a running time inferior to 5 seconds).}
	\label{fig:disc:methods:future:autoscore}
\end{figure}

% Because this software represents one of the very few open-source, free and exhaustive solution for sleep reading, scoring and analysis, it will probably reach in the near future a large public of sleep researchers, students and engineers, and hopefully lead to many great collaborations. These collaborations will be facilitated by the wide range of natively supported file formats, which allows sleep laboratories across the world to visualize and analyze their sleep data using a single, common software. In conclusion, we believe that the development of SLEEP represents a major step forward in sleep research, and more broadly in the emerging philosophy of open science.

Finally, it is also important to consider that SLEEP is part of a larger package, in which the author of the present thesis is a main contributor, entitled \href{http://visbrain.org/}{Visbrain}. Visbrain is a high-performance open source visualization suite dedicated to neuroscientific data at large. It currently includes 6 visualization modules, among which the three most prominent are SLEEP, BRAIN and SIGNAL. The BRAIN module is dedicated to EEG, magneto-encephalographic (MEG) and intra-cranial recordings, and allows notably the visualization of connectivity, sources and regions of interests on a 3D brain template. The SIGNAL module is dedicated to the visualization of uni- or multi-dimensional time-series arrays, and offers as such a convenient way to inspect datasets, locate artifacts and quickly analyze time-frequency properties of time-series. We are currently working on connecting the SIGNAL and the SLEEP module, notably through the visualization of automatically detected microstructural sleep events (e.g. spindles) inside the SIGNAL module. This will provide the users an interface to identify, at a glance, the false-positive events, and allow them to further investigate the time-frequency properties of each or all events.

%%%%%%%%%%%%%%%%%%%%%%%%%%%%%%%%%%%%%%%%%%%%%%%%%%%%%%%%%%%%%%%%%%%%%%%%%%%%%%%
\cleardoublepage
\chapter{General conclusion}
\label{disc:conclusion}

Throughout this work, we have addressed several unresolved issue related to the nature, the physiological correlates, and the function of dreaming. A large part of the present thesis was devoted to comparing cognitive and neurophysiological variables in high and low dream recallers, in an effort to understand \q{what cause sleepers sometimes dream, and sometimes do not} (Aristotle, On Sleep and Sleeplessness, 350 B.C., see section \ref{sec:dream-recall}). Our results revealed that the ability to recall dream is positively associated with (1) a longer duration of intra-sleep awakenings during sleep, (2) the strength of functional connectivity in specific areas of the brain during sleep, wakefulness, and notably the period following awakening and (3) creative-thinking abilities. Based on all these findings, we proposed a new model of the dream recall process integrating the contribution of all these factors as well as their interactions. A second aspect of our work was to better understand the relationship between waking life and dream content, through an exhaustive analysis of the characteristics of waking-life memories incorporated into dreams. Our findings, in addition with providing insights on the \emph{stuff that dreams are made on}, remarkably suggest the existence of a down-regulation of emotional waking memories during dreaming. Finally, we have been committed to developing a free and comprehensive software dedicated to sleep analysis, which will hopefully represent an important step forward in sleep research by providing students, researchers and engineers a common and portable platform for their analyses. Therefore, and although much works remain to be done, the present thesis opened up a new chapter in the understanding of this fascinating phenomenon that is dreaming. The theoretical and methodological contributions of the present work could serve as a basis for future research, in the hope that someday, we will be able to fully apprehend dreaming, in all its richness and diversity.
